\section{Tentative Solution} \label{sec:Tentive Soution}

To drop scalable video packets to retain streamed video quality when bandwidth is in sufficient and dynamic. We plan to implement the following three drop logics. (i) Tail, (ii) Enhancement Layer (EL), and (iii) Rate-Distortion Optimize (RDO). Tail always drop the last packet while EL drops the enhancement layer packets.  The advantage of tail is the simplicity. EL ensure the decodability since we always forward the base layer packet. RDO takes the nature of rates between packet length and distortion into consideration, and than drop the packet with largest rate. Furthermore, we will record the frame number and layer id of the dropped packet. Therefore, we can also drop the other packets with the same frame number and higher layer id. RDO logic aims to minimize the negative impact of dropping packets. 

\begin{figure}[tbh]
    \centering
    \includegraphics[width=0.40\textwidth]{fig/architecture.eps}
    \caption{High-level system architecture with a network of MANEs.}
\vspace{-0.1cm}
    \label{architecture} 
\end{figure}

Fig. ~\ref{architecture} shows our architecture of the whole system. It is composed of a sender, some clients, some P4-based MANEs, some regular switches, and a ONOS controller connects those MANEs. The ONOS controller forms a control plane disassociate from the data plane. The ONOS controller has the knowledge of the whole Internet. Therefore, it can optimally change the flow among those links. Before the streaming starts, ONOS controller will deploy some flows between each node in the topology. Once the streaming starts, those MANEs inside the topology starts to drop some packets if necessary. During the transmission, ONOS controller will keep monitoring the Internet condition and deploy flows in runtime. It may configure a new flow if one is broken and also notice the sender to let it adjust the sending bitrate if we encounter a network congestion. 