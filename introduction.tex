\section{Introduction} \label{sec:introduction}

In recent years, people are getting used to rely on Over-The-Top (OTT) 
services such as Skype, Facebook, Youtube, Netflix, etc. 
Among these services, video streaming is one of the services which 
consumes the most network resources. Video streaming needs more and more bandwidth because receivers prefer higher video quality than before, and thus incur high traffic amount on the best-effort Internet. Turns out streaming high quality video with less network resources becomes much more important.

{\em Scalable Video Coding (SVC) }is one of solutions for network congestion. Each of the SVC sequences contains a header, a base and multiple enhancement layers. The header stores the information that how the decoder should recognize those packets, and the base layer consists the elementary information of that particular frame. Enhancement layers are all rely on the low layers to decode. For example, enhancement layer 1 can be decode only if receiver receive the base layer and enhancement layer 2 can be decode only if receiver receive the enhancement layer 1. Furthermore, the encoder will encode the discardability into the packetize header. Therefore, We can drop the discardable packets without affecting its decodability in the middle-box of the Internet. The dynamic decisions on which video packets to drop can be sub-optimally done by streaming servers or clients without the global knowledge of the Internet. The better way to approach is through {\em Media-Aware Network Elements (MANEs)}, which are switches with knowledge of packets header. However, changing the normal switches into MANEs is quite difficult and thus not likely to happen. Fortunately, with recent advances in {\em Software-Defined Networking (SDN)} and {\em Network Function Virtualization (NVF)}, network switches are much more programmable, and make collaborative MANEs into reality. 

In the following sections, we will introduce every tools we are going to use and evaluate the results of our experiments among each algorithm we designed.