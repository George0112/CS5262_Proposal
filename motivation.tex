\section{Motivation} \label{sec:motivation}

In recent years, people are getting used to rely on Over-The-Top (OTT) 
services such as Skype, Facebook, video streaming, etc. 
Among these services, video streaming is one of the services which 
consumes the most network resources. Video streaming needs more and more bandwidth because receivers prefer higher video quality than before, and thus incur high traffic amount on the best-effort Internet. Turns out streaming high quality video with less network resources becomes much more important.

{\em Scalable Video Coding (SVC) }is one of solutions for network congestion. Each of the SVC sequences contains a base and multiple enhancement layers. Furthermore, the encoder will encode the discardability into the packetize header. Therefore, We can drop the discardable packets without affecting its decodability in the middle-box of the Internet.The dynamic decisions on which video packets to drop can be sub-optimally done by streaming servers or clients without the global knowledge of the Internet. The better way to approach is through {\em Media-Aware Network Elements (MANEs)}, which are switches with knowledge of packets header. However, changing the normal switches into MANEs is quite difficult and thus not likely to happen. Fortunately, with recent advances in {\em Software-Defined Networking (SDN)} and {\em Network Function Virtualization (NVF)}, network switches are much more programmable, and make collaborative MANEs into reality. 




\begin{comment}
Over-The-Top (OTT) video streaming services, like Apple TV, Netflix, and Hulu,
have become very popular, e.g., the streaming device market is projected to
exceed 25 billion USD by 2024, with a 15\%+ annual growth on unit shipments~\cite{ott_market}. These OTT services offer 4k resolution
video sequences, and thus incur high traffic amount on the
best-effort Internet. {\em Scalable Video Coding (SVC)} is a promising solution
to mitigate such excessive traffic: each scalable video
sequence contains a base and multiple enhancement layers, while higher
enhancement layers can be dropped without affecting its decodability.  
The dynamic
decisions on which video packets to
drop can be sub-optimally done by
streaming servers or clients without the global knowledge on the network.
A better way for making decisions is through {\em
Media-Aware Network Elements (MANEs)}, which are switches that have access to
the video-related packet headers and local network conditions, and thus can
make better decisions. By collaboratively considering multiple OTT video
streams sent across a set of interconnected MANEs, an even higher video
quality level is possible. 

Unfortunately, replacing regular switches with MANEs is quite expensive and therefore less likely to happen.
However, with recent advances in {\em Software-Defined Networking (SDN)} and
{\em Network Function Virtualization (NFV)}, network switches are much more
{\em programmable}, which in turn make collaborative MANEs into a reality. 
A recent market report points out that the
SDN/NFV investments from Internet Service Providers (ISPs) 
are expected to grow at an annual rate of 45\% between 2017 and 2020~\cite{sdn_market}. In
other words, programmable switches are going to be deployed in the Internet
scale, and {\em these switches can be programmed into collaborative MANEs for
optimally streaming scalable video sequences.}


In this work, we prototype the very first MANE in P4 programming
language~\cite{BDGI+14} for scalable video streaming. P4 is designed for
realizing different network protocols (in software) on packet processors that
can forward packets at the line speed. We implement intelligent media-aware
packet drop logics in packet processors of P4 switches, and we connect
multiple P4 switches to an ONOS controller through P4 runtime for a
network of collaborative MANEs. We demonstrate our prototype system using
mininet or a few real P4 switches to show its
practicality. Real scalable video sequences are created and used
in our demonstrations, while several demonstration scenarios are used to
show the merits of our system. 
\end{comment}
