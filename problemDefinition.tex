\section{Problem Definition} \label{sec:problemDefinition}

In modern network, we use Internet to deliver packets to exchange information all over the world. However, Internet is not designed for streaming videos or audios. The best-effort characteristic of Internet makes things become more complicated. Therefore, many experts had published many protocols and algorithms to make sure receiver can receive the whole data they are streaming and recover from packet loss. In our experiment, we use Scalable Video Evaluation Framework(SVEF) to stream our videos. In this framework, we stream videos based on UDP/IP network. In UDP protocol, sender will not receive any signal or ack to determine if receiver has already receive the packet or the packet is actually lost. With this characteristic, the sender can not know the condition of network or how to retransmit lost packets. In this situation, our goal is reduce or eliminate the undecodable packets or even select the packets with better quality and less packet size.

SVEF has already implemented the sender and receiver using socket programming in C language but it isn't using any further protocol such as RTP or RTSP to adjust sending rate or recover from packet loss. The sender of SVEF will first parse the trace file which we provide and build the packets of the whole video. In each video, header and base layer will be built into a single packet, and each enhancement layers will be pack into one packet. Then, it calculate the time gap between two frames as 1/fps. In this time gap, sender will try to send all the packets belong to that frame one by one and calculate the time it spend. If it costs less than the time gap, then it sleeps until the next time gap and transmit packets of next frame when time is up. On the other hand, sender will send an error message if it can not send all packets belong to that frame in the time gap. If that happens, it will cut the current transmission and start the next one.